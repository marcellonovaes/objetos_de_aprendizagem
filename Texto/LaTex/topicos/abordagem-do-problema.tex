O enriquecimento dos vídeos educacionais, da forma como é proposto neste trabalho, envolve uma série de questões desafiadoras:
\begin{enumerate}
    \item A detecção das lacunas semânticas é uma tarefa que depende da percepção dos estudantes;
    \item O conteúdo utilizado para preencher as lacunas deve satisfazer os estudantes;
    \item Mesmo vídeos curtos requerem muito esforço para serem analisados, pois existe muita informação concentrada em cada segmento;
    \item As atividades de geração de conteúdo complementar multimídia, e de agregação deste conteúdo ao vídeo, requerem uma quantidade grande de esforço e recursos;
\end{enumerate}


Para superar estas dificuldades foram selecionadas algumas técnicas que, em conjunto, permitem que o enriquecimento de vídeos seja feito de maneira colaborativa, e que os estudantes possam contribuir realizando apenas atividades que realmente necessitem de inteligência humana.

O paradigma de computação humana ajuda a modelar cenários nos quais parte das atividades devem ser realizadas por pessoas e a outra parte pode ser executada por meios automáticos \cite{VonAhn:2005:HC:1168246}. Sendo assim, este ponto de vista se mostra totalmente compatível com o enriquecimento de vídeos educativos, uma vez que as atividades de identificação das lacunas semânticas e obtenção das informações necessárias para cobri-las requerem inteligência humana, e a geração e agregação dos artefatos multimídia são atividades que podem ser automatizadas.

Um ponto positivo da computação humana é a possibilidade de paralelizar as atividades que requerem inteligência humana, desta forma é possível agilizar o processo de coleta das contribuições.  Todavia, para que este modelo seja aplicado neste trabalho é necessário que estas atividades sejam modeladas corretamente, e que sejam distribuídas de maneira eficiente entre os estudantes. Nestas condições, uma abordagem crowdsourcing baseada em micro-tarefas se mostra muito adequada para distribuir, coletar e gerenciar as tarefas e contribuições \cite{Difallah:2015:DMC:2736277.2741685}. Neste tipo de abordagem, uma série de pequenas tarefas é distribuída entre os colaboradores, gerando uma base de resultados parciais que são validados, filtrados e agregados, permitindo que se gere um resultado final a partir das contribuições.

Com base em trabalhos relacionados, uma maneira eficiente de modelar as micro-tarefas, em sistemas crowdsourcing que visam gerar artefatos sobre vídeos, é como tarefas de anotação \cite{ref:vidwiki2014,Wu:2011:VSV:1979742.1979803}. As anotações são meta-informações relacionadas com um artefato, e podem representar diferentes características dele tanto em relação à sua forma, quanto ao seu conteúdo \cite{Singhal:2014:GSA:2611040.2611056}. Neste projeto, as anotações sobre os vídeos serão utilizadas para representar as contribuições dos estudantes. Em outras palavras, para cada tarefa executada por um estudante será coletada uma anotação que irá relacionar a contribuição com o segmento de vídeo ao qual ela diz respeito. Este tipo de tarefa pode ser modelada e executada de forma muito simples, podendo ser realizada pelos estudantes enquanto assistem ao vídeo.

Uma vez que as contribuições sejam coletadas dos estudantes por meio das tarefas de anotação, elas são validadas e processadas de acordo com um conjunto de filtros e técnicas de agregação já consolidadas em sistemas crowdsourcing \cite{Alelyani:2016:SCR:2989238.2989245,Hipp:2013}. O resultado final do processo colaborativo é uma base consolidada de informações, que representam aquilo que deve ser adicionado em cada segmento do vídeo, para que as lacunas semânticas sejam preenchidas. Como estas informações são coletadas dos estudantes, este preenchimento tente a gerar resultados que são satisfatórios para eles.

Uma vez que se tenha estas informações, a geração dos artefatos multimídia, irão representar o conteúdo complementar que deve ser adicionado ao vídeo, é feita por meio de técnicas automáticas baseadas em modelos. Estas técnicas utilizam modelos pré-determinados para cada tipo de artefato que pode ser instanciado, e conforme os atributos de são preenchidos estes objetos multimídia são criados. Finalmente, uma vez que o conteúdo extra esteja pronto, ele é alinhado sobre o vídeo, e agregado a ele de acordo com suas características.
