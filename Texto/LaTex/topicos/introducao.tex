A utilização de vídeos como objetos de aprendizagem é uma prática já consolidada, e que cresce continuamente na medida em que as câmeras e \textit{smartphones} se tornam mais acessíveis, e as plataformas de distribuição como \textit{Youtube} e \textit{Vimeo} se popularizam  \cite{Davis:2015:YoutubeVimeo}. Todavia, o modelo clássico de produção de vídeos, que ainda é predominante atualmente, é um modelo centralizado que contempla apenas o ponto de vista do autor. O resultado deste processo focado em um ponto de vista único é a ocorrência de lacunas semânticas no vídeo \cite{Bhimani:2013:VPE:2465958.2465976}. 

Estas lacunas semânticas se caracterizam pela falta de informação necessária para que o estudante compreenda o conteúdo, com riqueza de detalhes. Elas ocorrem tanto nos casos em que o material didático realmente não oferece informação suficiente, quanto nos casos em que a maneira como a informação é apresentada não é adequada para aquele estudante. As lacunas semânticas em vídeos educacionais podem surgir por diversos motivos, e inevitavelmente geram problemas de compreensão, resultando em uma menor eficiência do vídeo como material didático. Neste trabalho serão abordados três tipos de lacunas semânticas: 
\begin{itemize}
    \item Termos e expressões de compreensão que podem não ser compreendidos;
    \item Conceitos que necessitam de explicações ou definições adicionais;
    \item Fatos e afirmações que precisam ser contextualizados.
\end{itemize}


Este trabalho apresenta um método para preencher as lacunas semânticas em vídeos educacionais por meio da agregação de artefatos multimídia que contenham informações adicionais. Este método também permite aprimorar a interação do estudante com o vídeo, adicionando recursos que melhorem a navegação e o acesso à informação contida neles. Existem diversos tipos de artefato multimídia e recursos interativos que podem ser agregados aos vídeos. Neste trabalho serão utilizados imagens, caixas de mensagem e hiperlinks.

O método proposto permite que o processo de enriquecimento, que consistem em gerar os artefatos multimídia e recursos interativos para então agrega-los ao vídeo, seja realizado sem a necessidade de profissionais experientes ou equipamentos e sistemas caros. Para tal, é utilizada uma abordagem \textit{Crowdsourcing} \cite{Howe2006} para coletar contribuições dos próprios estudantes, e determinar as informações usadas para gerar o conteúdo extra a ser agregado ao vídeo. Este método utiliza uma estratégia híbrida para gerar automaticamente o conteúdo extra, com base nas informações geradas a partir das contribuições dos estudantes.

O objetivo do método é gerar objetos de aprendizagem baseados em vídeos educacionais, que sejam mais eficientes como material didático que os vídeos originais. A questão de investigação deste trabalho, de forma complementar, é verificar se o método proposto pode realmente gerar objetos de aprendizagem que sejam mais eficazes que o vídeo original como material didático.

O restante do artigo apresenta-se da seguinte forma: a Seção 2 trata das lacunas semânticas em vídeos, suas causas e efeitos; na Seção 3 são justificadas e descritas a abordagem do problema, a estratégia de solução e as técnicas utilizadas; na Seção 4 é detalhado o método proposto; na Seção 5 é apresentado o ambiente de apoio ao método; na Seção 6 é apresentado o estudo de caso, com o experimento prático realizado e análise dos seus resultados; e finalmente, na Seção 7 são feitas as considerações finais.


