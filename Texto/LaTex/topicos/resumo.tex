Este artigo apresenta um método para a geração de objetos de aprendizagem com base no enriquecimento de vídeos. O método permite identificar lacunas de informação que podem ocorrer em vídeos educacionais, e preenche-las com conteúdo complementar por meio da agregação de artefatos multimídia, como imagens, caixas de texto, ou ainda recursos interativos como hiperlinks. O método é apoiado por um ambiente computacional, o que facilita a utilização de uma abordagem híbrida. Nesta abordagem, a identificação das lacunas semânticas, a obtenção de informação para gerar o conteúdo complementar, e a validação do conteúdo gerado, são realizados pelos próprios estudantes em uma abordagem crowdsourcing, enquanto os artefatos multimídia que são agregados aos vídeos são gerados por técnicas automáticas a partir das contribuições dos estudantes. O objetivo do trabalho é gerar objetos de aprendizagem baseados em vídeos educacionais, que sejam mais efetivos que os vídeos originas como material didático. A questão de investigação é determinar se os objetos de aprendizagem gerados são realmente materiais didáticos superiores ao vídeos originais, e esta verificação é feita por meio de um experimento prático que também é detalhado neste artigo.