This article introduces a method for generating learning objects based on video enrichment. The proposed method allows to identify information gaps that can occur in educational videos and fill them with complementary content by aggregating multimedia artifacts such as images, text boxes, or interactive features such as hyperlinks. This method is supported by a computational environment, which facilitates the use of a hybrid approach. In this approach, the identification of semantic gaps, the obtaining of information to generate the complementary content, and the validation of the generated content, are carried out by the students themselves in a crowdsourcing approach, while the multimedia artifacts that are added to the videos are generated by automatic techniques from the students' contributions. The goal of this work is to generate learning objects based on educational videos that are more effective than the original videos as didactic material. The research question is whether the learning objects generated are actually didactic material superior to the original videos, and this verification is done through the practical experience which is also detailed in this article.