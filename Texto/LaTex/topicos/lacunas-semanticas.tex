Lacunas semânticas são falhas de informação que dificultam a extração de informação útil do fluxo de vídeo este problema é tratado com frequência nos trabalhos relacionados com segmentação de imagens e identificação de objetos de cenas \cite{895972,SnoekICM2005}. Um exemplo clássico de lacuna semântica neste contexto é a oclusão de parte de um objeto: se em uma cena não se pode ser as mãos de um personagem, então não se pode identificar o que ele está segurando. Este mesmo conceito pode ser aplicado no contexto dos vídeos educativos, em um cenário no qual são os que estudantes precisam extrair informação útil do fluxo de vídeo.

As pessoas são capazes de preencher algumas lacunas semânticas por meio de inferências, associações com seu conhecimento prévio e o contexto identificado. Desta forma, cada indivíduo tem sua própria limitação em relação a quais lacunas é capaz de preencher. Isso significa que apesar de as lacunas semânticas contidas em um vídeo serem as mesmas, pessoas com diferentes características são afetadas por cada uma dela com intensidades diferentes.

No modelo clássico de produção de vídeo, que é utilizado de forma predominante, tudo é direcionado para o ponto de vista do autor. Isso significa que todas as cenas capturadas, conteúdo representado, e mesmo a forma como as informações são transmitidas, são determinados pelo que o autor quis mostrar e da forma como ele quis mostrar \cite{Bhimani:2013:VPE:2465958.2465976}. O problema neste processo é que o autor pode não perceber, no vídeo, as lacunas semânticas que ele é capaz de preencher, mas que podem causa sérios problemas de compreensão para outras pessoas que assistam ao vídeo. 

Estas lacunas semânticas em vídeos educacionais podem ocorrer sempre que um estudante não pode compreender parte do conteúdo por sentir falta de informação sobre ela. Elas podem acontecer mesmo quando a informação está presente no vídeo, caso ela não seja apresentada de uma forma que o estudante possa compreender, por exemplo, quando ele acessa um vídeo que está em um idioma que ele desconhece. 

São diversas as causas deste tipo de lacuna semântica, todavia foram selecionadas três delas para serem abordadas neste projeto. A primeira causa é ocorrência de expressões e termos que podem não ser compreendidos corretamente pelo estudante. Este problema está diretamente relacionado com os regionalismos e termos técnicos. A segunda causa diz respeito aos conceitos que são apresentados ao estudante, mas que precisam se melhor explicados ou exemplificados para que ele os compreenda. A terceira causa é a falta de contextualização ao se apresentar um fato. Isso ocorre quando se apresenta uma afirmação ao estudante, mas não se diz em quais condições aquele fato ocorreu, ou é válido.
